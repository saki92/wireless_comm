

\documentclass[journal]{IEEEtran}
\usepackage{cite}
\usepackage{bm}
\usepackage{amsmath}
\usepackage{amssymb}
%\usepackage[LGRgreek]{mathastext}
\begin{document}

\title{Performance Evaluation of Algorithm for Maximizing Sum-Rate in Multiuser MISO System with Perfect and Imperfect CSIT}


\author{Sakthivel~Velumani, Rijan~Kusatha% <-this % stops a space
}


\maketitle


\begin{abstract}
%\boldmath
This paper studies the performance and results of sum-rate (SR) maximization algorithm proposed in \cite{main}. We evaluated the algorithm for Multi User Multiple Input Single Output (MU-MISO) Broadcast (BC) system with 2 and 4 users. We have considered different Channel State Information at Transmitter (CSIT) scenarios as described in \cite{prelim} and \cite{main}. The objective of this paper is to evaluate the performance of algorithm that maximizes the SR or weighted SR (weights can be thought as a prioritizing factor) under perfect CSIT and maximizes Ergodic sum-rate (ESR) under imperfect CSIT. Numerical simulations show the ESR gains for different downlink channel SNRs and feedback channel qualities.
\end{abstract}

\IEEEpeerreviewmaketitle



\section{Introduction}

\IEEEPARstart{T}{he} use of multiple antennas at the transmitter and many single antenna receivers have proved to increase of overall throughput of a wireless communication system by exploiting spacial multiplexing capability. This technique is also called as transmit beamforming. The simple transmit beamforming techniques such as Zero Forcing Beamforming (ZF-BF) and Minimum Mean Square Error (MMSE) beamforming do not maximize the sum-rate but minimize the correlated and un-correlated noise. 
\subsection{Contributions}
\cite{prelim} proposes an iterative algorithm to find the optimum transmit filter that maximizes the sum-rate. In this algorithm, a Weighted Minimum Mean Square Error (WMMSE) matrix is calculated as an alternative way to find the optimum Weighted Sum-Rate (WSR). In each iteration three steps are involved in finding MMSE receive filter matrix, weight matrix (in context to WMMSE) and finally the updated transmit filter matrix.This is repeated till convergence. This algorithm assumes a perfect CSIT exists. 
\par In a wireless communication system, channel estimation is done at the receiver which is mostly assumed to be perfect. This CSI is fed back to the transmitter through a different uplink channel in case of FDD or if TDD is employed the estimation is done at transmitter, but at different time of the same channel. This raises doubts about the reliability of the CSI at the transmitter and hence we are forced to consider scenarios with imperfect CSIT.
\par To combat this uncertainaty in the CSIT, \cite{main} proposes an iterative algorithm that maximizes the ESR with the help of Average Sum-Rate (AVR). This algorithm is similar to the one in \cite{prelim} but the key difference is in each iteration, the cost function is built on an average of many channel samples. More details are presented in \ref{imper_alg}
\subsection{Organization}
In this paper we have evaluated the performance and compared the results of both the algorithms for different channel and system scenarios. The structure of rest of the paper is as follows. Next section tells about the system model we had considered for the evaluation of the algorithm followed by section \ref{per_alg} and \ref{imper_alg} that describes the details of both algorithms. Finally the results and conclusion are presented.
%\hfill mds

%\hfill January 11, 2007

\section{System Model} \label{sys_mod}
Consider a MU-MIMO system where the Base Station (BS) has $N_t$ antennas and $K$ users with single antenna each, such that $K \, \leq \, N_t$ and $\mathcal{K} \triangleq \{1,..., K\}$. The received signal at $k$th user is given as
\begin{equation} 
 y_k  =  \bm{\mathrm{h}}^H_k \bm{\mathrm{x}} + n_k
\end{equation}
where $\bm{\mathrm{h}}^H_k \in \mathbb{C}^{N_t}$ is the channel vector between BS and the $k$th user, $\bm{\mathrm{x}} \in \mathbb{C}^{N_t}$ is the transmit signal vector, and $n_k \sim \mathcal{CN}(0,\sigma^2_{n,k})$ is the Additive White Gaussian Noise (AWGN) at the $k$th user. The transmitted signal has a power constraint defined by E$\{\bm{\mathrm{x}}^H\bm{\mathrm{x}}\}\leq P_t$ and we assume equal noise variances across all users, i.e $\sigma^2_{n,k} = \sigma^2_n\ ,\ \ \forall k \in \mathcal{K}$. By definition the transmit SNR can be written as SNR $\triangleq P_t/\sigma^2_n$.

\subsection{CSIT Knowledge}
The channel state is given by $\bm{\mathrm{H}} \triangleq [\bm{\mathrm{h}}_1,...,\bm{\mathrm{h}}_K]$ that varies according to an ergodic stationary process with probability density $f_\mathrm{H}(\bm{\mathrm{H}})$. The receivers are assumed to have a perfect CSIR and transmitter are assumed to have an imperfect CSIT of the downlink channel. So the channel state estimate at transmitter is given by $\widehat{\bm{\mathrm{H}}} \triangleq [\widehat{\bm{\mathrm{h}}}_1,...,\widehat{\bm{\mathrm{h}}}_K]$, the error in the estimate is given as $\widetilde{\bm{\mathrm{H}}} \triangleq [\widetilde{\bm{\mathrm{h}}}_1,...,\widetilde{\bm{\mathrm{h}}}_K]$ and hence the overall channel relation can be written as $\bm{\mathrm{H}} = \widehat{\bm{\mathrm{H}}}+\widetilde{\bm{\mathrm{H}}}$. However the magnitude of $\widetilde{\bm{\mathrm{H}}}$ is allowed to vary w.r.t SNR by assuming that the feedback bits increase or decrease w.r.t downlink SNR. The gradient of this change is represented by an $\alpha$ factor and the error variance can be defined by $\sigma^2_{e,k} = P^{-\alpha}_t$

\subsection{Precoder and Transmit Signal Model}
At a particular channel usage, lets consider the symbol streams be $\bm{\mathrm{s}} \triangleq [s_1,...,s_K]$ for $K$ users, and are mapped to the transmit antennas through a precoding matrix $\bm{\mathrm{P}}_p \triangleq [\bm{\mathrm{p}}_1,...,\bm{\mathrm{p}}_K]$, where $\bm{\mathrm{p}}_k \in \mathbb{C}^{N_t}$ is the $k$th user's precoder. This yields a transmit signal model $\bm{\mathrm{x}} = \sum_{k=1}^K \bm{\mathrm{p}}_k s_k$. Assuming the data symbol distribution as E$\{\bm{\mathrm{ss}}^H\} = \bm{\mathrm{I}}$, the transmit power constraints boils down to E$\{\bm{\mathrm{x}}^H\bm{\mathrm{x}}\} = \text{tr}(\bm{\mathrm{PP}}^H) \leq P_t$.

\section{Algorithm with Perfect CSIT} \label{per_alg}

\section{Algorithm with Imperfect CSIT} \label{imper_alg}

\section{Results} \label{res}
% An example of a floating figure using the graphicx package.
% Note that \label must occur AFTER (or within) \caption.
% For figures, \caption should occur after the \includegraphics.
% Note that IEEEtran v1.7 and later has special internal code that
% is designed to preserve the operation of \label within \caption
% even when the captionsoff option is in effect. However, because
% of issues like this, it may be the safest practice to put all your
% \label just after \caption rather than within \caption{}.
%
% Reminder: the "draftcls" or "draftclsnofoot", not "draft", class
% option should be used if it is desired that the figures are to be
% displayed while in draft mode.
%
%\begin{figure}[!t]
%\centering
%\includegraphics[width=2.5in]{myfigure}
% where an .eps filename suffix will be assumed under latex, 
% and a .pdf suffix will be assumed for pdflatex; or what has been declared
% via \DeclareGraphicsExtensions.
%\caption{Simulation Results}
%\label{fig_sim}
%\end{figure}

% Note that IEEE typically puts floats only at the top, even when this
% results in a large percentage of a column being occupied by floats.


% An example of a double column floating figure using two subfigures.
% (The subfig.sty package must be loaded for this to work.)
% The subfigure \label commands are set within each subfloat command, the
% \label for the overall figure must come after \caption.
% \hfil must be used as a separator to get equal spacing.
% The subfigure.sty package works much the same way, except \subfigure is
% used instead of \subfloat.
%
%\begin{figure*}[!t]
%\centerline{\subfloat[Case I]\includegraphics[width=2.5in]{subfigcase1}%
%\label{fig_first_case}}
%\hfil
%\subfloat[Case II]{\includegraphics[width=2.5in]{subfigcase2}%
%\label{fig_second_case}}}
%\caption{Simulation results}
%\label{fig_sim}
%\end{figure*}
%
% Note that often IEEE papers with subfigures do not employ subfigure
% captions (using the optional argument to \subfloat), but instead will
% reference/describe all of them (a), (b), etc., within the main caption.


% An example of a floating table. Note that, for IEEE style tables, the 
% \caption command should come BEFORE the table. Table text will default to
% \footnotesize as IEEE normally uses this smaller font for tables.
% The \label must come after \caption as always.
%
%\begin{table}[!t]
%% increase table row spacing, adjust to taste
%\renewcommand{\arraystretch}{1.3}
% if using array.sty, it might be a good idea to tweak the value of
% \extrarowheight as needed to properly center the text within the cells
%\caption{An Example of a Table}
%\label{table_example}
%\centering
%% Some packages, such as MDW tools, offer better commands for making tables
%% than the plain LaTeX2e tabular which is used here.
%\begin{tabular}{|c||c|}
%\hline
%One & Two\\
%\hline
%Three & Four\\
%\hline
%\end{tabular}
%\end{table}


% Note that IEEE does not put floats in the very first column - or typically
% anywhere on the first page for that matter. Also, in-text middle ("here")
% positioning is not used. Most IEEE journals use top floats exclusively.
% Note that, LaTeX2e, unlike IEEE journals, places footnotes above bottom
% floats. This can be corrected via the \fnbelowfloat command of the
% stfloats package.



\section{Conclusion}
The conclusion goes here.





% if have a single appendix:
%\appendix[Proof of the Zonklar Equations]
% or
%\appendix  % for no appendix heading
% do not use \section anymore after \appendix, only \section*
% is possibly needed

% use appendices with more than one appendix
% then use \section to start each appendix
% you must declare a \section before using any
% \subsection or using \label (\appendices by itself
% starts a section numbered zero.)
%


\appendices
\section{Proof of the First Zonklar Equation}
Appendix one text goes here.

% you can choose not to have a title for an appendix
% if you want by leaving the argument blank
\section{}
Appendix two text goes here.


% use section* for acknowledgement
\section*{Acknowledgment}
We thank 
\ifCLASSOPTIONcaptionsoff
  \newpage
\fi

\begin{thebibliography}{1}

\bibitem{prelim} 
S\o{}ren Skovgaard Christensen, Rajiv Agarwal, Elisabeth de Carvalho, and John M. Cioffi 
\textit{Weighted Sum-Rate Maximization using Weighted MMSE for
MIMO-BC Beamforming Design}. 
IEEE TRANSACTIONS ON WIRELESS COMMUNICATIONS, VOL. 7, NO. 12, DECEMBER 2008

\bibitem{main} 
Hamdi Joudeh and Bruno Clerckx 
\textit{Sum-Rate Maximization for Linearly Precoded Downlink Multiuser MISO Systems With Partial CSIT: A Rate-Splitting Approach}. 
IEEE TRANSACTIONS ON COMMUNICATIONS, VOL. 64, NO. 11, NOVEMBER 2016.
\end{thebibliography}




\end{document}


